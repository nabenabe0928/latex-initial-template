%%%%%%%%%%%%%%%%%%%%%%%%%%%%%%%%%%%%%%%%%%%%%%%%%%%%%%%
%%%% DEFINE CONFERENCE SPECIFIC PACKAGES AND STYLE %%%%
%%%%%%%%%%%%%%%%%%%%%%%%%%%%%%%%%%%%%%%%%%%%%%%%%%%%%%%

\usepackage[utf8]{inputenc}
\usepackage{amsmath}
\usepackage{amssymb}
\usepackage{mathtools}
\usepackage{bm}
\usepackage{bbm}
\usepackage{tikz}
\usepackage{ascmac}
\usepackage{fancyhdr}
\usepackage{graphicx}
\usepackage{algorithm, algpseudocode}
\usepackage{algpseudocode}
\usepackage{ulem}
\usepackage{booktabs}
\usepackage{multirow}
\usepackage[caption=false]{subfig}
\usepackage{comment}
\usepackage{listings}
\usepackage{lastpage}
\usepackage{tcolorbox}
\usepackage{multibib}
\tcbuselibrary{breakable, skins, theorems}

% a mathematical statement that is proved using rigorous mathematical reasoning.
% In a mathematical paper, the term theorem is often reserved for the most important results.
\newtheorem{theorem}{Theorem}

% a result in which the (usually short) proof relies heavily on a given theorem.
\newtheorem{corollary}{Corollary}

% a proved and often interesting result, but generally less important than a theorem.
\newtheorem{proposition}{Proposition}

% a minor result whose sole purpose is to help in proving a theorem.
% It is a stepping stone on the path to proving a theorem.
% Very occasionally lemmas can take on a life of their own.
\newtheorem{lemma}{Lemma}

% an assertion that is then proved.
% It is often used like an informal lemma.
\newtheorem{claim}{Claim}

% a precise and unambiguous description of the meaning of a mathematical term.
% It characterizes the meaning of a word by giving all the properties and only those properties that must be true.
\newtheorem{definition}{Definition}

% a statement that is assumed to be true without proof.
% These are the basic building blocks from which all theorems are proved.
\newtheorem{assumption}{Assumption}

\newtheorem{proof}{Proof}

\def\qed{\hfill $\Box$}

% remove the end from algorithm
\algtext*{EndFor}
\algtext*{EndWhile}
\algtext*{EndIf}
\algtext*{EndProcedure}
\algtext*{EndFunction}
\algnewcommand{\LineComment}[1]{\State \(\triangleright\) #1}
\newcommand{\Break}{\textbf{break}}
\newcommand{\Continue}{\textbf{continue}}

% Add `References` for Appendix
\newcites{appx}{References}
